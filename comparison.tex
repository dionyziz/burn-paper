\section{Comparison}

We present a few alternatives for performing proof of burn which have been proposed in previous work. All of the schemes presented are captured by our burn protocol definition.

\newcommand{\opreturn}{\texttt{OP\_RETURN}}

\noindent
\textbf{$\opreturn$.}
Bitcoin natively employs a mechanism for burning money using a specialized
opcode called $\opreturn$~\cite{bartoletti2017analysis}. Unfortunately,
creating an $\opreturn$ proof-of-burn transaction requires specialized software and thus is not
user-friendly. However, it does benefit the Bitcoin network by allowing the UTXO
to be pruned, at the cost of not being uncensorable.

\noindent
\textbf{P2SH $\opreturn$.}
An $\opreturn$ script can also be used as the redeemScript for a P2SH~\cite{p2sh} address. The script is hashed and turned into an address. The scheme is unspendable since there is no scriptSig that could make an $\opreturn$ script succeed. Additionally it is uncensorable if the tag is not revealed. Finally, the scheme is user-friendly since the user can create a burn transaction using any regular wallet.

\noindent
\textbf{Nothing-up-my-sleeve address.}
An address is manually crafted, such that it is obvious that it has not been obtained by regular keypair generation through the secure signature scheme's \textsf{Gen}. For example, a public key hash of all zeros could be considered a nothing-up-my-sleeve address. The public key hash is then converted to an address through the regular procedure of the cryptocurrency. The premise is that it would be very hard for someone to obtain the public and secret key corresponding to this public key hash, thus funds sent to the corresponding address are considered unspendable.

We compare the aforementioned schemes on whether they satisfy the burn protocol properties we define: Binding, unspendability and uncensorability. Additionally, we compare them based on how easily they translate to multiple cryptocurrencies. For instance, $\opreturn$ and P2SH $\opreturn$ rely on Bitcoin Script semantics and do not directly apply to any non-Bitcoin based cryptocurrencies like Ethereum, thus we say they are not flexible. The comparison is illustrated on the table that follows.

\begin{center}
    \newcommand{\y}{$\bullet$}
    \newcommand{\n}{}
    \begin{tabular}{ |c|c|c|c|c| }
     \hline
                                        & Binding & Flexible & Unspendable & Uncensorable \\
     \hline
     $\opreturn$                         & \y      & \n       & \y          & \n \\
     P2SH $\opreturn$                    & \y      & \n       & \y          & \y \\
     Nothing-up-my-sleeve address       & \n      & \y       & \y          & \y \\
     $a \xor 1$ \textbf{(this work)}    & \y      & \y       & \y          & \y \\
     \hline
    \end{tabular}
\end{center}
