\section{Construction}\label{section:construction}

We outline the blockchain address protocol for Bitcoin pay to public key hash (P2PKH) transactions. The implementation is parameterized by a secure signature scheme $S$, and a hash function $H$.
$\GenAddress$ uses $S$ to generate a keypair, and hashes the public key in order to generate the public key hash. A tuple consisting of the public key hash and the secret key is returned.
$\SpendVerify$ takes a spending transaction $m$, a signature $\sigma$ and a public key hash $pkh$. The signature should contain the public key $pk$ corresponding to $pkh$ such that $H(pk) = pkh$ and a valid signature for the outpoint $m$, $\sigma'$. If these conditions are met the function returns true, otherwise it returns false.
The implementation can be seen in Algorithm~\ref{alg.bitcoin}.

\import{./}{algorithms/alg.bitcoin.tex}

\todo{explain the bitcoin protocol here ~gtklocker}

Bitcoin and Bitcoin Cash employ mechanisms for burning money using a specialized
opcode called \textsf{OP\_RETURN}~\cite{bartoletti2017analysis}. Unfortunately,
creating an \text{OP\_RETURN} proof-of-burn transaction is complicated and not
user-friendly. However, it does benefit the Bitcoin network by allowing the UTXO
to be pruned.

As we wish our proof-of-burn mechanism to be user friendly, we propose a
different burn protocol. Our protocol appears in
Algorithm~\ref{alg.construction}.

\import{./}{algorithms/alg.construction.tex}
